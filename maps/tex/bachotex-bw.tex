\setupbodyfont[antykwa-poltawskiego]

\starttext

\startluacode

    local root = xml.load("bachotex.osm")

    local coordinates = { }

    local b = xml.first(root,"/osm/bounds")

    local minlat = b.at.minlat
    local minlon = b.at.minlon
    local maxlat = b.at.maxlat
    local maxlon = b.at.maxlon
    local midlat = 0.5 * (minlat + maxlat)

    local deg_to_rad = math.pi / 180.0

    -- vertical scale: 1" = 1cm
    local scale = 3600

    local function f_pair(lon, lat)
        return string.formatters("(%.2fcm,%.2fcm)", (lon - minlon) * scale * math.cos(midlat * deg_to_rad), (lat-minlat) * scale)
    end
    -- this only works for positive numbers
    local function f_degree_to_str(num)
        local deg = math.floor(num)
        num = (num - deg) * 60
        local min = math.floor(num)
        num = (num - min) * 60
        local sec = num
        return string.formatters("%d°%d’%.0f”", deg, min, sec)
    end

    for c in xml.collected(root,"/osm/node") do
        local a = c.at
        coordinates[a.id] = a
    end

    local f_pattern = string.formatters["/osm/(way|relation)[@visible='true']/tag[@k='%s']"]
    local f_draw    = string.formatters["draw %--t withcolor %s withpen pencircle scaled 1;"]
    local f_fill    = string.formatters["fill %--t -- cycle withcolor %s withpen pencircle scaled 1;"]
    local f_draw_p  = string.formatters["path p ; p := %--t ; draw p withcolor %s withpen pencircle scaled 1;"]
    local f_fill_p  = string.formatters["path p ; p := %--t -- cycle ; fill p withcolor %s withpen pencircle scaled 1;"]
    local f_bounds  = string.formatters["setbounds currentpicture to %--t -- cycle ; addbackground withcolor %s ;"]
    local f_way     = string.formatters["/osm/way[@id='%s']"]
    local f_textext = string.formatters['draw (textext("\\bf %s") scaled 0.35) shifted center p withcolor white;']

    local f_draw_p_dashed = string.formatters["path p ; p := %--t ; draw p withcolor %s dashed %s withpen pencircle scaled %g;"]
    local f_label_urt     = string.formatters['label.urt((textext("%s") scaled 0.35), llcorner p) withcolor white;']

    -- we can use transparencies to add up

    local colors = {
        restaurant        = "white",
        parking           = ".25white",
        university        = ".2white", -- MC, background
        beach             = ".1white", -- MC
        water             = "black",   -- MC
--
        path              = ".4white", -- MC
        track             = ".4white", -- MC
        footway           = ".3white", -- MC
        service           = ".4white", -- MC
        steps             = ".3white", -- MC, steps
--
        hut               = "white",   -- MC
        yes               = "white",   -- MC
        public            = "white",   -- MC
        fence             = ".5white",   -- MC
        pitch             = ".3white",
        playground        = ".3white",
        pier              = ".3white",
        reservoir_covered = ".2white", -- MC
        background        = ".3white", -- MC, grass
--
        grid              = "white",
    }

    local rendering = {
        natural  = true,
        highway  = false,
        building = true,
        barrier  = false,
        leisure  = true,
        amenity  = true,
        landuse  = false,
        man_made = false,
    }

    local shown = {
        "amenity",
        "natural",
        "highway",
        "building",
        "barrier",
        "leisure",
        "man_made",
    }


    local function drawshapes(what)
        local function filterpath(r,pattern,name,kind)
            local p = { }
            for c in xml.collected(r,pattern) do
                local coordinate = coordinates[c.at.ref]
                if coordinate then
                    p[#p+1] = f_pair(coordinate.lon, coordinate.lat)
                end
            end
            local color = colors[kind] or "black"
            if #p == 0 then
                -- error
            elseif name then
                if rendering[what] then
                    context(f_fill_p(p,color))
                else
                    context(f_draw_p(p,color))
                end
                context(f_textext(name))
            else
                if rendering[what] then
                    context(f_fill(p,color))
                else
                    context(f_draw(p,color))
                end
            end
        end

         -- we're in tex so filters print to tex (maybe make that an option some day) which is why we need the xml://

        for c in xml.collected(root,f_pattern(what)) do
            local parent = xml.parent(c)
            local tag    = parent.tg
            local name   = xml.filter(parent,"xml://tag[@k='addr:housenumber']/attribute('v')")
            local kind   = xml.filter(parent,"xml://tag[@k='amenity']/attribute('v')") or c.at.v
            if tag == "way" then
                filterpath(parent,"/nd",name,kind)
            elseif tag == "relation" then
                for m in xml.collected(parent,"/member[@type='way']") do
                    local f = xml.first(root,f_way(m.at.ref))
                    if f then
                        filterpath(f,"/nd",name,kind)
                    end
                end
            end
        end
    end

    local function draw_grid()
        local lat0 = math.ceil (3600*minlat)
        local lat1 = math.floor(3600*maxlat)
        local lat
        for i=lat0,lat1,1 do
            lat=i/3600
            local p = {
                f_pair(minlon,lat),
                f_pair(maxlon,lat),
            }
            context(f_draw_p_dashed(p,colors.grid, "withdots", 0.4))
            context(f_label_urt(f_degree_to_str(lat)))
        end

        local lon0 = math.ceil (1800*minlon)*2
        local lon1 = math.floor(1800*maxlon)*2
        local lon
        for i=lon0,lon1,2 do
            lon=i/3600
            local p = {
                f_pair(lon, minlat),
                f_pair(lon, maxlat),
            }
            context(f_draw_p_dashed(p,colors.grid, "withdots", 0.4))
            context(f_label_urt(f_degree_to_str(lon)))
        end
    end

    local function drawboundary()
        local p = {
            f_pair(minlon,minlat),
            f_pair(maxlon,minlat),
            f_pair(maxlon,maxlat),
            f_pair(minlon,maxlat),
        }
        context(f_bounds(p,colors.background))
    end

    context.startTEXpage()
    context.startMPcode("doublefun")

    for i=1,#shown do
        drawshapes(shown[i])
    end

    --draw_grid()

    drawboundary()

    context.stopMPcode()
    context.stopTEXpage()

\stopluacode

% height: math.sin((53.2908400-53.2870100)*3.14/180)*6400000 = 427.60 m
% width: 303.57
\stoptext
